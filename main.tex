\documentclass[10pt, a4paper, titlepage]{article}

\usepackage[T2A]{fontenc}
\usepackage[utf8]{inputenc}
\usepackage[russian]{babel}


\usepackage[affil-it]{authblk} % Принадлежность на титульном листе

\usepackage{amsmath} % Система уравнений

\usepackage{pgfplots} % Графики
\pgfplotsset{width=10cm,compat=1.9}
\usepgfplotslibrary{external}
\tikzexternalize

\usepackage{float} % Вставить картинку в определенное место


\title{Алгоритмы численных методов. \\ Самостоятельные работы}
\author{Верхотуров В.С.}
\affil{БСБО-05-20}
\affil{РТУ МИРЭА}
\date\today


\begin{document}

\maketitle
\tableofcontents
\newpage

\section{Самостоятельная работа \textnumero 1}

\subsection{Задача 1}

Дано уравнение $5x^2+2x-6=0$.

\subsubsection*{Точное решение уравнения}
$$D=b^2-4ac=2^2-4\times5\times(-6)=4+120=124,$$
$$x=\frac{-b\pm\sqrt{D}}{2a},$$
\begin{equation*}
    \begin{cases}
        $$x_1=\frac{-2+\sqrt{124}}{2\times5}$$, \\
        $$x_2=\frac{-2-\sqrt{124}}{2\times5}$$,
    \end{cases}
    \Longleftrightarrow
    \begin{cases}
        $$x_1=-0,2+0,2\sqrt{31}$$, \\
        $$x_2=-0,2-0,2\sqrt{31}$$.
    \end{cases}
\end{equation*} 

\subsubsection*{График левой части уравнения}
\begin{figure}[H]
    \centering
    \begin{tikzpicture}
        \begin{axis}[axis lines = left, grid = both, xlabel = \(x\), ylabel = {\(f(x)\)}]
            \addplot [domain=-2:2, samples=30, color=red]{x^2*5 + 2*x - 6};
            \addlegendentry{\(5x^2 + 2x - 6\)}
        \end{axis}
    \end{tikzpicture}
    \caption{График функции}
    \label{sr1Task1Function}
\end{figure}

\subsubsection*{Приближенное значение левого корня методом половинного деления}
Необходимо найти приближенное значение левого корня $5x^2+2x-6=0$ методом половинного деления с точностью $\varepsilon=10^{-3}$.

По графику на рис.~\ref{sr1Task1Function} можно предположить, что значение левого корня находится на интервале $[-1,5;-1]$. $f(-1,5)=2,25$, $f(-1)=-3$. $f(x_0)f(x_1)=2,25\times(-3)=-6,75\leq0$, следовательно промежуток действительно содержит хотя бы один корень. Для первой итерации: $x_0=-1,5$, $x_1=-1$.

$$x_2=\frac{1}{2}(x_0+x_1)$$

\begin{center}
    \begin{tabular}{|c|c|c|c|c|c|c|c|}
        \hline
        $x_0$ & $x_1$ & $x_2$ & $f(x_0)$ & $f(x_1)$ & $f(x_2)$ & $f(x_0)f(x_2)$ & $f(x_1)f(x_2)$ \\ \hline
        
        -1.5 & -1 & -1.25 &	2.25 & -3 & -0.6875 & -1.546875 & 2.0625 \\ \hline
        
        -1.5 & -1.25 & -1.375 &	2.25 &	-0.6875	& 0.7031 & 1.581975 & -0.48338125 \\ \hline
        
        -1.25 &	-1.375 & -1.3125 & -0.6875 & 0.7031 & -0.0117 &	0.00804375 & -0.00822627 \\ \hline
        
        -1.375 & -1.3125 & -1.3438 & 0.7031 & -0.0117 &	0.3414 & 0.24003834 & -0.00399438 \\ \hline
        
        -1.3125 & -1.3438 & -1.3282 & -0.0117 &	0.3414 & 0.1642	& -0.00192114 &	0.05605788 \\ \hline
        
        -1.3125 & -1.3282 & -1.3204 & -0.0117 & 0.1642 & 0.0765 & -0.00089505 & 0.0125613 \\ \hline
        
        -1.3125 & -1.3204 & -1.3165 & -0.0117 & 0.0765 & 0.0329 & -0.00038493 & 0.00251685 \\ \hline
        
        -1.3125 & -1.3165 & -1.3145 & -0.0117 & 0.0329 & 0.0106 & -0.00012402 & 0.00034874 \\ \hline
        
        -1.3125 & -1.3145 & -1.3135 & -0.0117 & 0.0106 & -0.0006 & 0.00000702 & -0.00000636 \\ \hline
        
    \end{tabular}
\end{center}

Функции электронной таблицы\footnote{Google Sheets}:
\begin{itemize}
    \item для 1 столбца, кроме первой итерации --- $=IF(G2<=0, A2, B2)$\footnote{для клетки A3};
    
    \item для 2 столбца, кроме первой итерации --- $=C2$\footnote{для клетки B3};
    
    \item для 3 столбца --- $=ROUND((A3+B3)/2, 4)$\footnote{для клетки C3};
    
    \item для 4 столбца --- $=ROUND(POWER(A3, 2)*5+2*A3-6, 4)$\footnote{для клетки D3};
    
    \item для 5 столбца --- $=ROUND(POWER(B3, 2)*5+2*B3-6, 4)$\footnote{для клетки E3};
    
    \item для 6 столбца --- $=ROUND(POWER(C3, 2)*5+2*C3-6, 4)$\footnote{для клетки F3};
    
    \item для 7 столбца --- $=D3*F3$\footnote{для клетки G3};
    
    \item для 8 столбца --- $=E3*F3$\footnote{для клетки H3}.

\end{itemize}

Приближенным значением левого корня при $\varepsilon=10^{-3}$ является $x_2$ при 9 итерации, равная -1,3135.

\subsubsection*{Приближенное значение правого корня методом простой итерации}

Необходимо найти приближенное значение правого корня методом простой итерации с точностью $\varepsilon=10^{-6}$.

%В соответствии с критерием сходимости наибольшая скорость сходимости обеспечивается при $|\varphi'(x)|=0$.
%$$f(x)=0,$$
%$$f(x)=5x^2+2x-6,$$
%$$x=\varphi(x),$$
%$$x=\frac{-6-5x^2}{2}=-3-2{,}5x^2,$$
%$$\varphi(x)=-3-2{,}5x^2,$$
%$$\varphi'(x)=-5x,$$
%$$|-5x|=0,$$
%$$x=x_0=0.$$
По рис.~\ref{sr1Task1Function} начальное приближение к правому корню $x_0=0$.

Итерационная формула: $$x_{k+1}=x_k-\frac{f(x_k)}{f'(x_k)}.$$

Нахождение производной $f(x)$:
$$f(x)=5x^2+2x-6,$$
$$f'(x)=10x+2.$$

\begin{center}
    \begin{tabular}{|c|c|c|c|}
        \hline
        $x_k$ & $f(x_k)$ & $f'(x_k)$ & $x_{k+1}$ \\ \hline
        
        0 & -6 & 2 & 3 \\ \hline
        
        3 &	45 & 32 & 1.59375 \\ \hline
        
        1.59375	& 9.887695313 &	17.9375 & 1.042519599 \\ \hline
        
        1.042519599 & 1.519274773 & 12.42519599 & 0.9202458924 \\ \hline
        
        0.9202458924 & 0.07475429702 & 11.20245892 & 0.9135728666 \\ \hline
        
        0.9135728666 & 0.0002226463642 & 11.13572867 & 0.9135528727 \\ \hline
        
        0.9135528727 & 0.000000001998776256	& 11.13552873 & 0.9135528726 \\ \hline
        
        0.9135528726 & 0 & 11.13552873 & 0.9135528726 \\ \hline
        
    \end{tabular}
\end{center}

Функции электронной таблицы\footnote{Google Sheets}:
\begin{itemize}
    \item для 1 столбца, кроме первой итерации --- $=D2$\footnote{для клетки A3};
    
    \item для 2 столбца --- $=POWER(A3, 2)*5+2*A3-6$\footnote{для клетки B3};
    
    \item для 3 столбца --- $=10*A3+2$\footnote{для клетки C3};
    
    \item для 4 столбца --- $=A3-B3/C3$\footnote{для клетки D3}.
\end{itemize}

Критерий окончания процесса $|x_7-x_6|=|0,9135528726-0,9135528727|=10^{-10}<\varepsilon=10^{-6}$ выполнен. 

Приближенное значение правого корня равно 0,9135529 при $\varepsilon=10^{-6}$.

\subsection{Задача 2}

Дано уравнение $x^2\exp(x)-6=0$.

\subsubsection*{График левой части уравнения}
\begin{figure}[H]
    \centering
    \begin{tikzpicture}
        \begin{axis}[axis lines = left, grid = both, xlabel = \(x\), ylabel = {\(f(x)\)}]
            \addplot [domain=-5:2, samples=100, color=red]{x^2*exp(x)-6};
            \addlegendentry{\(x^2\exp(x)-6\)}
        \end{axis}
    \end{tikzpicture}
    \caption{График функции}
    \label{sr1Task2Function}
\end{figure}

\subsubsection*{Нахождение приближенного решения}

Необходимо найти приближенное решение методом простой итерации при $\varepsilon=10^{-6}$.

По рис.~\ref{sr1Task2Function} начальное приближение к корню $x_0=1$.

Нахождение производной $f(x)$: $$f(x)=x^2\exp(x)-6,$$ $$f'(x)=2x\exp(x)+x^2\exp(x).$$

\begin{center}
    \begin{tabular}{|c|c|c|c|}
        \hline
        $x_k$ & $f(x_k)$ & $f'(x_k)$ & $x_{k+1}$ \\ \hline
        
        1 & -3.281718172 & 8.154845485 & 1.402425549 \\ \hline
        
        1.402425549 & 1.995125904 & 19.39698023 & 1.299567997 \\ \hline
        
        1.299567997 & 0.1943141286 & 15.72719605 & 1.287212703 \\ \hline
        
        1.287212703 & 0.002470136078 & 15.32877669 & 1.287051559 \\ \hline
        
        1.287051559 & 0.0000004141451742 & 15.32363686 & 1.287051532 \\ \hline
        
        1.287051532 & 0 & 15.323636 & 1.287051532 \\ \hline
        
    \end{tabular}
\end{center}

Функции электронной таблицы\footnote{Google Sheets}:
\begin{itemize}
    \item для 1 столбца, кроме первой итерации --- $=D2$\footnote{для клетки A3};
    
    \item для 2 столбца --- $=POWER(A3,2)*EXP(A3)-6$\footnote{для клетки B3};
    
    \item для 3 столбца --- $=2*A3*EXP(A3)+POWER(A3,2)*EXP(A3)$\footnote{для клетки C3};
    
    \item для 4 столбца --- $=A3-B3/C3$\footnote{для клетки D3}.
\end{itemize}

Критерий окончания процесса $|x_5-x_4|=|1,287051559-1,287051532 |=2,7\times10^{-8}<\varepsilon=10^{-6}$ выполнен. 

Приближенное значение корня равно 1,2870515 при $\varepsilon=10^{-6}$.

\subsection{Задача 3}

Дана система уравнений:
\begin{equation*}
    \begin{cases}
        $$2x_1+6x_2-x_3=-12+6$$ \\
        $$5x_1-x_2+2x_3=29+6$$ \\
        $$-3x_1-4x_2+x_3=5+6$$
    \end{cases}
    \Longleftrightarrow
    \begin{cases}
        $$2x_1+6x_2-x_3=-6$$ \\
        $$5x_1-x_2+2x_3=35$$ \\
        $$-3x_1-4x_2+x_3=11$$
    \end{cases}
\end{equation*}


\subsubsection*{Точное решение системы уравнений}

Решение СЛАУ методом Гаусса. Прямой ход:
\begin{equation*}
    \begin{pmatrix}
        2 & 6 & -1 & \vrule & -6 \\
        5 & -1 & 2 & \vrule & 35 \\
        -3 & -4 & 1 & \vrule & 11
    \end{pmatrix}
\end{equation*}

\begin{equation*}
    \begin{pmatrix}
        0 & 32 & -9 & \vrule & -100 \\
        5 & -1 & 2 & \vrule & 35 \\
        -3 & -4 & 1 & \vrule & 11
    \end{pmatrix}
\end{equation*}

\begin{equation*}
    \begin{pmatrix}
        0 & 32 & -9 & \vrule & -100 \\
        0 & -23 & 11 & \vrule & 160 \\
        -3 & -4 & 1 & \vrule & 11
    \end{pmatrix}
\end{equation*}

\begin{equation*}
    \begin{pmatrix}
        0 & 0 & 146 & \vrule & 2820 \\
        0 & -23 & 11 & \vrule & 160 \\
        -3 & -4 & 1 & \vrule & 11
    \end{pmatrix}
\end{equation*}

%$$x_1=\frac{-6-6x_2+x_3}{2}=-3-3x_2+\frac{1}{2}x_3,$$
%\begin{multline*}
    %x_2=\frac{35-5x_1-2x_3}{-1}=-35+5x_1+2x_3=-35+5\left(-3-3x_2+\frac{1}{2}x_3\right)+2x_3= \\
    %=-35-15-15x_2+\frac{5}{2}x_3+2x_3=-50-15x_2+\frac{9}{2}x_3,
%\end{multline*}
%$$16x_2=-50+\frac{9}{2}x_3,$$
%$$x_2=-\frac{50}{16}+\frac{9}{32}x_3,$$
%$$x_1=-3-3\left(-\frac{50}{16}+\frac{9}{32}x_3\right)+\frac{1}{2}x_3=-3+\frac{75}{8}-\frac{27}{32}x_3+\frac{1}{2}x_3=\frac{51}{8}-\frac{11}{32}x_3,$$
%\begin{multline*}
    %x_3=11+3x_1+4x_2=11+3\left(\frac{51}{8}-\frac{11}{32}x_3\right)+4\left(-\frac{50}{16}+\frac{9}{32}x_3\right)= \\
    %=11+\frac{153}{8}-\frac{33}{32}x_3-\frac{25}{2}+\frac{9}{8}x_3=\frac{141}{8}+\frac{3}{32}x_3,
%\end{multline*}
%$$\frac{29}{32}x_3=\frac{141}{8}.$$

Обратный ход:

$$x_3=\frac{2820}{145}=\frac{564}{29},$$
$$x_2=\frac{160-11x_3}{-23}=\frac{160-11\times\frac{564}{29}}{-23}=\frac{68}{29},$$
$$x_1=\frac{11+4\times_2-x_3}{-3}=\frac{11+4\times\frac{68}{29}-\frac{568}{29}}{-3}=-\frac{9}{29}.$$
%$$x_3=\frac{564}{29},$$
%$$x_2=-\frac{50}{16}+\frac{9}{32}\times\frac{564}{29}=\frac{68}{29},$$
%$$x_1=\frac{51}{8}-\frac{11}{32}\times\frac{564}{29}=-\frac{9}{29}.$$

\subsubsection*{Приближенное решение системы уравнений}

Решение СЛАУ методом Зейделя.

\begin{equation*}
    \begin{cases}
        $$x_1=7+\frac{1}{5}x_1+\frac{1}{6}x_3$$, \\
        $$x_2=-1-\frac{1}{3}x_1+\frac{1}{6}x_3$$, \\
        $$x_3=11+3x_1+4x_2$$.
    \end{cases}
\end{equation*}

Примем за начальное приближение:
\begin{equation*}
    \begin{cases}
        $$x_1^{\{0\}}=0$$, \\
        $$x_2^{\{0\}}=0$$, \\
        $$x_3^{\{0\}}=0$$.
    \end{cases}
\end{equation*}

\begin{center}
    \begin{tabular}{|c|c|c|}
        \hline
        $x_1^{\{m\}}$ & $x_2^{\{m\}}$ & $x_3^{\{m\}}$ \\ \hline
        
        0 & 0 & 0 \\ \hline
        
        7 & -3.333333333 & 18.66666667 \\ \hline
        
        -1.133333333 & 2.488888889 & 17.55555556 \\ \hline
        
        0.4755555556 & 1.767407407 & 19.4962963 \\ \hline
        
        -0.445037037 & 2.397728395 &19.25580247 \\ \hline
        
        -0.2227753086 & 2.283558848 & 19.46590947 \\ \hline
        
        -0.3296520165 & 2.35420225 & 19.42785295 \\ \hline
        
        -0.3003007298 & 2.338075735 & 19.45140075 \\ \hline
        
        -0.312945153 & 2.346215176 & 19.44602524 \\ \hline
        
        -0.3091670628 & 2.344059895 & 19.44873839 \\ \hline
        
        -0.3106833778 & 2.345017525 & 19.44801997 \\ \hline
        
        -0.3102044811 & 2.344738155 & 19.44833917 \\ \hline
        
        -0.310388039 & 2.344852542 & 19.44824605 \\ \hline
        
        -0.3103279122 & 2.344816979 & 19.44828418 \\ \hline
        
    \end{tabular}
\end{center}

Функции электронной таблицы\footnote{Google Sheets}:
\begin{itemize}
    \item для 1 столбца, кроме первой итерации --- $=7+B2/5-C2*2/5$\footnote{для клетки A3};
    
    \item для 2 столбца, кроме первой итерации  --- $=-1-A3/3+C2/6$\footnote{для клетки B3};
    
    \item для 3 столбца, кроме первой итерации --- $=11+3*A3+4*B3$\footnote{для клетки C3}.
    
\end{itemize}

$$|x_1^{\{13\}}-x_1^{\{12\}}|<10^{-4}$$
$$|x_2^{\{13\}}-x_2^{\{12\}}|<10^{-4}$$
$$|x_3^{\{13\}}-x_3^{\{12\}}|<10^{-4}$$

При $\varepsilon=10^{-4}$:
\begin{equation*}
    \begin{cases}
        $$x_1=0,31032$$, \\
        $$x_2=2,34481$$, \\
        $$x_3=19,44828$$.
    \end{cases}
\end{equation*}




\end{document}
