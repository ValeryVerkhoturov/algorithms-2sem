\documentclass[10pt, a4paper, titlepage]{article}

\usepackage[T2A]{fontenc}
\usepackage[utf8]{inputenc}
\usepackage[russian]{babel}


\usepackage[affil-it]{authblk} % Принадлежность на титульном листе

\usepackage{amsmath} % Система уравнений

\usepackage{pgfplots} % Графики
\pgfplotsset{width=10cm,compat=1.9,}
\usepgfplotslibrary{external}
\tikzexternalize


\title{Алгоритмы численных методов \\ Самостоятельные работы}
\author{Верхотуров В.С.}
\affil{БСБО-05-20}
\affil{РТУ МИРЭА}
\date\today

\begin{document}

\maketitle

\section{Самостоятельная работа \textnumero 1}

\subsection{Задача 1}

Дано уравнение $5x^2+2x-6=0$.

\subsubsection{Точное решение уравнения}
$$D=b^2-4ac=2^2-4\times5\times(-6)=4+120=124,$$
$$x=\frac{-b\pm\sqrt{D}}{2a},$$
\begin{equation*}
    \begin{cases}
        $$x_1=\frac{-2+\sqrt{124}}{2\times5}=-0.2+0.2\sqrt{31},$$ \\
        $$x_2=\frac{-2-\sqrt{124}}{2\times5}=-0.2-0.2\sqrt{31}.$$
    \end{cases}
\end{equation*} 

\subsubsection{График левой части уравнения}
\begin{figure}[h]
    \centering
    \begin{tikzpicture}
        \begin{axis}[axis lines = left, grid = both, xlabel = \(x\), ylabel = {\(f(x)\)}]
            \addplot [domain=-2:2, samples=30, color=red]{x^2*5 + 2*x - 6};
            \addlegendentry{\(5x^2 + 2x - 6\)}
        \end{axis}
    \end{tikzpicture}
    \caption{График функции}
    \label{task1Function}
\end{figure}

\subsubsection{Приближенное значение левого корня методом половинного деления}
Необходимо найти приближенное значение левого корня $5x^2+2x-6=0$ методом половинного деления с точностью $\varepsilon=10^{-3}$.

По графику на рис.~\ref{task1Function} можно предположить, что значение левого корня находится на интервале $[-1{,}5;-1]$. $f(-1{,}5)=2{,}25$, $f(-1)=-3$. $f(x_0)f(x_1)=2{,}25\times(-3)=-6{,}75\leq0$, следовательно промежуток действительно содержит хотя бы один корень.

$$x_2=\frac{1}{2}(x_0+x_1)$$

Дано для итерации 1: $x_0=-1{,}5$, $x_1=-1$. 

\newcounter{iteration}
\setcounter{iteration}{1} 
\begin{center}
    \begin{tabular}{|c|c|c|c|c|c|c|c|c|}
        \hline
        N & $x_0$ & $x_1$ & $x_2$ & $f(x_0)$ & $f(x_1)$ & $f(x_2)$ & $f(x_0)f(x_2)$ & $f(x_1)f(x_2)$ \\ \hline
        
        \arabic{iteration}\stepcounter{iteration} & -1.5 &	-1 &	-1.25 &	2.25 &	-3	& -0.6875 &	-1.546875 &	2.0625 \\ \hline
        
        \arabic{iteration}\stepcounter{iteration} & -1.5 &	-1.25 &	-1.375 &	2.25 &	-0.6875	& 0.7031 &	1.581975 &	-0.48338125 \\ \hline
        
        \arabic{iteration}\stepcounter{iteration} & -1.25 &	-1.375 &	-1.3125	& -0.6875 &	0.7031 &	-0.0117 &	0.00804375	& -0.00822627 \\ \hline
        
        \arabic{iteration}\stepcounter{iteration} & -1.375 &	-1.3125 &	-1.3438	& 0.7031 &	-0.0117 &	0.3414 &	0.24003834	& -0.00399438 \\ \hline
        
        \arabic{iteration}\stepcounter{iteration} & -1.3125 &	-1.3438	& -1.3282 &	-0.0117 &	0.3414 &	0.1642	& -0.00192114 &	0.05605788 \\ \hline
        
        \arabic{iteration}\stepcounter{iteration} & -1.3125 &	-1.3282 &	-1.3204	& -0.0117 &	0.1642	& 0.0765 &	-0.00089505	& 0.0125613 \\ \hline
        
        \arabic{iteration}\stepcounter{iteration} & -1.3125 &	-1.3204	& -1.3165 &	-0.0117	& 0.0765	& 0.0329	& -0.00038493 &	0.00251685 \\ \hline
        
        \arabic{iteration}\stepcounter{iteration} & -1.3125 &	-1.3165	& -1.3145 &	-0.0117	& 0.0329 &	0.0106 &	-0.00012402	& 0.00034874 \\ \hline
        
        \arabic{iteration}\stepcounter{iteration} & -1.3125 &	-1.3145	& -1.3135 &	-0.0117	& 0.0106 &	-0.0006	& 0.00000702 &	-0.00000636 \\ \hline
        
    \end{tabular}
\end{center}

Функции Excel таблицы:
\begin{itemize}
    \item для 2 столбца, кроме первой итерации --- $A3=IF(G2<=0, A2, B2)$;
    
    \item для 3 столбца, кроме первой итерации --- $B3=C2$;
    
    \item для 4 столбца --- $C3=ROUND((A3+B3)/2, 4)$;
    
    \item для 5 столбца --- $D4=ROUND(POWER(A3, 2)*5+2*A3-6, 4)$;
    
    \item для 6 столбца --- $E3=ROUND(POWER(B3, 2)*5+2*B3-6, 4)$;
    
    \item для 7 столбца --- $F3=ROUND(POWER(C3, 2)*5+2*C3-6, 4)$;
    
    \item для 8 столбца --- $G3=D3*F3$;
    
    \item для 9 столбца --- $H3=E3*F3$.

\end{itemize}

Приближенным значением левого корня при $\varepsilon = 10^{-3}$ является $x_2$ при 9 итерации и равен -1,3135.

\subsubsection{Приближенное значение правого корня методом простой итерации}

Необходимо найти приближенное значение правого корня методом простой итерации с точностью $\varepsilon=10^{-6}$.

Итерационная формула: $$x_{k+1}=x_k-\frac{f(x_k)}{f'(x_k)}$$

Дано для первой итерации: $x_0=0$.

Нахождение производной:
$$f(x)=5x^2+2x-6,$$
$$f'(x)=10x+2.$$

\begin{center}
    \begin{tabular}{|c|c|c|c|}
        \hline
        $x_k$ & $f(x_k)$ & $f'(x_k)$ & $x_{k+1}$ \\ \hline
        
        0 & -6 & 2 & 3 \\ \hline
        
        3 &	45 & 32 & 1.59375 \\ \hline
        
        1.59375	& 9.887695313 &	17.9375 & 1.042519599 \\ \hline
        
        1.042519599 & 1.519274773 & 12.42519599 &	0.9202458924 \\ \hline
        
        0.9202458924 &	0.07475429702 &	11.20245892 & 0.9135728666 \\ \hline
        
        0.9135728666 & 0.0002226463642	& 11.13572867 & 0.9135528727 \\ \hline
        
        0.9135528727 & 0.000000001998776256	& 11.13552873	& 0.9135528726 \\ \hline
        
        0.9135528726 &	0 & 11.13552873 & 0.9135528726 \\ \hline
        
    \end{tabular}
\end{center}

Функции Excel таблицы:
\begin{itemize}
    \item для 1 столбца, кроме итерации 1 --- $A3=D2$;
    
    \item для 2 столбца --- $B3=POWER(A3, 2)*5+2*A3-6$;
    
    \item для 3 столбца --- $C3=10*A3+2$;
    
    \item для 4 столбца --- $D3=A3-B3/C3$;

\end{itemize}

Критерий окончания процесса $|x_8-x_7|=|0{,}9135528726-0{,}9135528727|=10^{-10}<\varepsilon=10^{-6}$ выполнен. 

Приближенное значение правого корня равно 0,9135528726.

\subsection{Задача 2}

Дано уравнение $x^2e^x-6=0$.

\subsubsection{График левой части уравнения}
\begin{figure}[htbp]
    \centering
    \begin{tikzpicture}
        \begin{axis}[axis lines = left, grid = both, xlabel = \(x\), ylabel = {\(f(x)\)}]
            \addplot [domain=-10:10, samples=100, color=red]{x^2*exp(x)-6};
            \addlegendentry{\(x^2e^x-6\)}
        \end{axis}
    \end{tikzpicture}
    \caption{График функции}
\end{figure}





\end{document}
