% !TeX encoding = UTF-8
% !TeX spellcheck = russian-aot
% !TeX program = pdflatex

\documentclass[10pt, a4paper, titlepage]{article}

\usepackage[T2A]{fontenc}
\usepackage[utf8]{inputenc}
\usepackage[russian]{babel}

\usepackage{authblk} % Принадлежность на титульном листе

\usepackage{amsmath} % Система уравнений

\usepackage{mathtools} % Multlined

\usepackage{pgfplots} % Графики
\pgfplotsset{width=.8\textwidth, compat=1.9}
%\usepgfplotslibrary{external}
%\tikzexternalize

\usepackage{icomma} % Убрать расстояние у запятой между цифрами

\usepackage{gauss} % Визуализация операций над матрицей

\usepackage{array}

\usepackage{hyperref}
\hypersetup{pdftitle={Алгоритмы численных методов. Самостоятельные работы}, pdfauthor={В. С. Верхотуров}, colorlinks=false, pdfborder={0 0 0}}

\title{Алгоритмы численных методов.~Самостоятельные работы}
\author{В.~С.~Верхотуров}
\affil{БСБО-05-20}
\affil{РТУ МИРЭА}
\date\today

\newcommand*\hm[1]{#1\nobreak\discretionary{}{\hbox{$\mathsurround=0pt #1$}}{}} % Дублирование математического оператора при переносе выражения на новую строку

\righthyphenmin=2 % Допуск переноса слога в 2 буквы

% Столбцы с математикой
\newcolumntype{C}{>{$}c<{$}}

\usepackage{siunitx} % форматирование чисел
\sisetup{output-decimal-marker={,}, round-mode=figures, round-precision=3, round-pad=false, drop-zero-decimal}

\usepackage{booktabs}


\begin{document}

\maketitle
\tableofcontents
\newpage

\section*{Задание}

\addcontentsline{toc}{section}{Задание}

\subsection*{Самостоятельная работа \textnumero~1. Решение нелинейных уравнений и систем линейных уравнений}

\subsubsection*{Задача~1}

Дано уравнение $5x^2+2x-6=0$.

\begin{enumerate}
    \item Найти точное решение уравнения;
    \item построить график левой части этого уравнения;
    \item найти приближенное значение левого корня этого уравнения методом половинного деления с~точностью $\varepsilon=10^{-3}$;
    \item найти приближенное значение правого корня методом простой итерации, $\varepsilon=10^{-6}$.
\end{enumerate}

\subsubsection*{Задача~2}

Дано уравнение $x^2\exp(x)-6=0$.

\begin{enumerate}
    \item Построить график левой части уравнения;
    \item найти приближенное решение методом простой итерации, $\varepsilon=10^{-6}$.
\end{enumerate}

\subsubsection*{Задача 3}

Найти точное решение системы уравнений:
\begin{equation*}
    \begin{cases}
        $$2x_1+6x_2-x_3 = -12+6$$, \\
        $$5x_1-x_2+2x_3 = 29+6$$, \\
        $$-3x_1-4x_2+x_3 = 5+6$$.
    \end{cases}
\end{equation*}

\subsection*{Самостоятельная работа \textnumero~2. Построение интерполяционных многочленов}

\begin{enumerate}
    \item Найти приближение функции, заданной в~точках, многочленом, значения которого совпадают со~значениями функции в~указанных точках:
    \begin{center}
    	\begin{tabular}{SC}
    		\toprule
    		
    		$x$ & $y$ \\
    		
    		\midrule
    		
    		1 & 0+6 \\
    		3 & 4+6 \\
    		5 & 2+6 \\
    		7 & 6+6 \\
    		9 & 8+6 \\
    		
    		\bottomrule
    	\end{tabular}
    \end{center}
    \item построить график полученного интерполяционного многочлена;
    \item найти значение функции в~точке $x=6$.
\end{enumerate}

\subsection*{Самостоятельная работа \textnumero~3. Вычисление определённых интегралов}

\begin{enumerate}
    \item Найти аналитическое выражение для~неопределённого интеграла $$\int \sin(\ln(6x))\,dx ;$$
    \item построить графики найденного интеграла~--- красным цветом и подынтегральной функции~--- синим цветом;
    \item вычислить приближенное значение этого интеграла: $$\int\limits_2^{6+2} \sin(\ln(6x))\,dx ;$$
    \item вычислить приближенное значение интеграла $$\int\limits_2^{6+2} \exp(-x^2)\sin(6x)\,dx .$$
\end{enumerate}

\subsection*{Самостоятельная работа \textnumero~4. Решение обыкновенных дифференциальных уравнений}

\begin{enumerate}
    \item Найти аналитическое решение задачи Коши: $$y'(t)=\frac{1}{6}(t+y), \quad y(0)=6 ;$$
    \item построить график найденного решения на~отрезке $[0; 6]$;
    \item найти численное решение задачи Коши: $$y'(t)=\frac{1}{6}(t+y), \quad y(0)=6 ;$$
    \item найти численное решение задачи Коши $$y'(t)=\sin(6y(t)+t^2), \quad y(0)=6$$ в точках $t=1$ и $t=2$;
    \item построить график найденного решения на~отрезке $[0;\,5]$.
\end{enumerate}

\clearpage
\section{Самостоятельная работа \textnumero~1. Решение нелинейных уравнений и систем линейных уравнений}

\subsection{Задача~1}

Дано уравнение $5x^2+2x-6=0$.

\subsubsection*{Точное решение уравнения}

\begin{align*}
D &= b^2-4ac=2^2-4\times5\times(-6)=4+120=124, \\
x &= \frac{-b\pm\sqrt{D}}{2a},
\end{align*}
\begin{equation*}
    \begin{cases}
        $$x_1=\frac{-2+\sqrt{124}}{2\times5}$$, \\
        $$x_2=\frac{-2-\sqrt{124}}{2\times5}$$,
    \end{cases}
    \Longleftrightarrow
    \begin{cases}
        $$x_1=-0,2+0,2\sqrt{31}$$, \\
        $$x_2=-0,2-0,2\sqrt{31}$$.
    \end{cases}
\end{equation*} 

\subsubsection*{График левой части уравнения}

См.~рис.~\ref{sr1Task1Function} на~стр.~\pageref{sr1Task1Function}.

\begin{figure}[htb]
    \centering
    \begin{tikzpicture}
        \begin{axis}[legend pos = north west, axis lines = left, grid = both, xlabel = $x$, ylabel = {$f(x)$}, ymax=11]
            \addplot [domain=-2:2, samples=30, color=red]{x^2*5 + 2*x - 6};
            \addlegendentry{\(5x^2 + 2x - 6\)};
        \end{axis}
    \end{tikzpicture}
    \caption{График функции}
    \label{sr1Task1Function}
\end{figure}

\subsubsection*{Приближенное значение левого корня методом половинного деления}

Необходимо найти приближенное значение левого корня $5x^2+2x-6=0$ методом половинного деления с~точностью $\varepsilon=10^{-3}$.

По~графику на~рис.~\ref{sr1Task1Function} можно предположить, что значение левого корня находится на~интервале $[-1,5;-1]$. $f(-1,5)=2,25$, $f(-1)=-3$. $f(x_0)f(x_1) \hm =2,25\times(-3)=-6,75\leq0$, следовательно промежуток действительно содержит хотя бы один корень. Для первой итерации: $x_0=-1,5$, $x_1=-1$.

Функция нахождения $x_2$:
$$x_2=\frac{1}{2}(x_0+x_1).$$

См.~таблицу~\ref{sr1Task1Table} на~стр.~\pageref{sr1Task1Table}.

\begin{table}[htb]
	\centering
	\resizebox{\textwidth}{!}{
		\begin{tabular}{|*{8}{S|}}
			\toprule
			
			{$x_0$} & {$x_1$} & {$x_2$} & {$f(x_0)$} & {$f(x_1)$} & {$f(x_2)$} & {$f(x_0)f(x_2)$} & {$f(x_1)f(x_2)$} \\
			
			\midrule
			
			-1.5 & -1 & -1.25 &	2.25 & -3 & -0.6875 & -1.546875 & 2.0625 \\
			
			-1.5 & -1.25 & -1.375 & 2.25 &	-0.6875	& 0.7031 & 1.581975 & -0.48338125 \\
			
			-1.25 & -1.375 & -1.3125 & -0.6875 & 0.7031 & -0.0117 & 0.00804375 & -0.00822627 \\
			
			-1.375 & -1.3125 & -1.3438 & 0.7031 & -0.0117 &	0.3414 & 0.24003834 & -0.00399438 \\
			
			-1.3125 & -1.3438 & -1.3282 & -0.0117 & 0.3414 & 0.1642 & -0.00192114 &	0.05605788 \\
			
			-1.3125 & -1.3282 & -1.3204 & -0.0117 & 0.1642 & 0.0765 & -0.00089505 & 0.0125613 \\
			
			-1.3125 & -1.3204 & -1.3165 & -0.0117 & 0.0765 & 0.0329 & -0.00038493 & 0.00251685 \\
			
			-1.3125 & -1.3165 & -1.3145 & -0.0117 & 0.0329 & 0.0106 & -0.00012402 & 0.00034874 \\
			
			-1.3125 & -1.3145 & -1.3135 & -0.0117 & 0.0106 & -0.0006 & 0.00000702 & -0.00000636 \\
			
			\bottomrule
		\end{tabular}
	}
	\caption{Результаты приближения методом половинного деления}
	\label{sr1Task1Table}
\end{table}

Функции\footnote{представлены формулы для~строки~3} электронной таблицы\footnote{Google Sheets}~\ref{sr1Task1Table}:
\begin{description}
    \item[столбец A, кроме первой итерации] \verb"=IF(G2<=0, A2, B2)";
    
    \item[столбец B, кроме первой итерации] \verb"=C2";
    
    \item[столбец C] \verb"=ROUND((A3+B3)/2, 4)";
    
    \item[столбец D] \verb"=ROUND(POWER(A3, 2)*5+2*A3-6, 4)";
    
    \item[столбец E] \verb"=ROUND(POWER(B3, 2)*5+2*B3-6, 4)";
    
    \item[столбец F] \verb"=ROUND(POWER(C3, 2)*5+2*C3-6, 4)";
    
    \item[столбец G] \verb"=D3*F3";
    
    \item[столбец H] \verb"=E3*F3".

\end{description}

Приближенным значением левого корня при~$\varepsilon=10^{-3}$ является $x_2$ при~$9$ итерации, равная $-1,3135$.

\subsubsection*{Приближенное значение правого корня методом простой итерации}

Необходимо найти приближенное значение правого корня методом простой итерации с~точностью $\varepsilon=10^{-6}$.

По~рис.~\ref{sr1Task1Function} начальное приближение к~правому корню $x_0=0$.

Итерационная формула: $$x_{k+1}=x_k-\frac{f(x_k)}{f'(x_k)} .$$

Нахождение производной $f(x)$:
\begin{align*}
    f(x) &= 5x^2+2x-6, \\
    f'(x) &= 10x+2.
\end{align*}

См.~таблицу~\ref{sr1Task1Table2} на~стр.~\pageref{sr1Task1Table2}.

\begin{table}[htb]
	\centering
	\begin{tabular}{|*{4}{S|}}
		\toprule
		
		{$x_k$} & {$f(x_k)$} & {$f'(x_k)$} & {$x_{k+1}$} \\
		
		\midrule
		
		0 & -6 & 2 & 3 \\
		
		3 &	45 & 32 & 1,59375 \\
		
		1,59375	& 9,887695313 &	17,9375 & 1,042519599 \\
		
		1,042519599 & 1,519274773 & 12,42519599 & 0,9202458924 \\
		
		0,9202458924 & 0,07475429702 & 11,20245892 & 0,9135728666 \\
		
		0,9135728666 & 0,0002226463642 & 11,13572867 & 0,9135528727 \\
		
		0,9135528727 & 0,000000001998776256	& 11,13552873 & 0,9135528726 \\
		
		0,9135528726 & 0 & 11,13552873 & 0,9135528726 \\
		
		\bottomrule
	\end{tabular}
	\caption{Результаты приближения методом простой итерации}
	\label{sr1Task1Table2}
\end{table}

Функции\footnote{представлены формулы для~строки 3} электронной таблицы\footnote{Google Sheets}~\ref{sr1Task1Table2}:
\begin{description}
    \item[столбец A, кроме первой итерации] \verb"=D2";
    
    \item[столбец B] \verb"=POWER(A3, 2)*5+2*A3-6";
    
    \item[столбец C] \verb"=10*A3+2";
    
    \item[столбец D] \verb"=A3-B3/C3".
\end{description}

Критерий окончания процесса $|x_7-x_6|=|0,9135528726-0,9135528727| \hm =10^{-10}<\varepsilon=10^{-6}$ выполнен. 

Приближенное значение правого корня равно 0,9135529 при~$\varepsilon=10^{-6}$.

\subsection{Задача~2}

Дано уравнение $x^2\exp(x)-6=0$.

См.~рис.~\ref{sr1Task2Function} на~стр.~\pageref{sr1Task2Function}.

\subsubsection*{График левой части уравнения}

\begin{figure}[htb]
    \centering
    \begin{tikzpicture}
        \begin{axis}[legend pos = north west, axis lines = left, grid = both, xlabel = $x$, ylabel = {$f(x)$}, ymax=-1, xmax=2, xmin=-4]
            \addplot [domain=-5:2, samples=100, color=red]{x^2*exp(x)-6};
            \addlegendentry{\(x^2\exp(x)-6\)};
        \end{axis}
    \end{tikzpicture}
    \caption{График функции}
    \label{sr1Task2Function}
\end{figure}

\subsubsection*{Нахождение приближенного решения}

Необходимо найти приближенное решение методом простой итерации при~$\varepsilon=10^{-6}$.

По~рис.~\ref{sr1Task2Function} начальное приближение к~корню $x_0=1$.

Нахождение производной $f(x)$:
\begin{align*}
f(x) &= x^2\exp(x)-6, \\ 
f'(x) &= 2x\exp(x)+x^2\exp(x).
\end{align*}

См.~таблицу~\ref{sr1Task2Table} на~стр.~\pageref{sr1Task2Table}.

\begin{table}[htb]
    \centering
    \begin{tabular}{|*{4}{S[round-precision=5]|}}
        \toprule
        
        {$x_k$} & {$f(x_k)$} & {$f'(x_k)$} & {$x_{k+1}$} \\
        
        \midrule
        
        1 & -3,281718172 & 8,154845485 & 1,402425549 \\
        
        1,402425549 & 1,995125904 & 19,39698023 & 1,299567997 \\
        
        1,299567997 & 0,1943141286 & 15,72719605 & 1,287212703 \\
        
        1,287212703 & 0,002470136078 & 15,32877669 & 1,287051559 \\
        
        1,287051559 & 0,0000004141451742 & 15,32363686 & 1,287051532 \\
        
        1,287051532 & 0 & 15,323636 & 1,287051532 \\
        
        \bottomrule  
    \end{tabular}
    \caption{Результаты приближения методом простой итерации}
    \label{sr1Task2Table}
\end{table}

Функции\footnote{представлены формулы для~строки 3} электронной таблицы\footnote{Google Sheets}~\ref{sr1Task2Table}:
\begin{description}
    \item[столбец A, кроме первой итерации] \verb"=D2";
    
    \item[столбец B] \verb"=POWER(A3,2)*EXP(A3)-6";
    
    \item[столбец C] \verb"=2*A3*EXP(A3)+POWER(A3,2)*EXP(A3)";
    
    \item[столбец D] \verb"=A3-B3/C3".
\end{description}

Критерий окончания процесса $|x_5-x_4|=|1,287051559-1,287051532 | \hm =2,7\times10^{-8}<\varepsilon=10^{-6}$ выполнен. 

Приближенное значение корня равно 1,2870515 при~$\varepsilon=10^{-6}$.

\subsection{Задача 3}

Дана система уравнений:

\begin{equation*}
    \begin{cases}
        $$2x_1+6x_2-x_3 = -12+6$$ \\
        $$5x_1-x_2+2x_3 = 29+6$$ \\
        $$-3x_1-4x_2+x_3 = 5+6$$
    \end{cases}
    \Longleftrightarrow
    \left\{
        \begin{tabular}{@{\hskip0pt} >{$\phantom{}}r<{x_1$} @{\hskip0pt} >{$\phantom{}}r<{x_2$} @{\hskip0pt} >{$\phantom{}}r<{x_3$} @{\hskip0pt} >{$\phantom{}}c<{\phantom{}$} @{\hskip0pt} C}
            \phantom{+}2 & +6 & -\phantom{1} & = & -6 \\
            \phantom{+}5 & -\phantom{1} & +2 & = & \phantom{+}35 \\
            -3 & -4 & +\phantom{1} & = & \phantom{+}11
        \end{tabular}
    \right.
\end{equation*}


\subsubsection*{Точное решение системы уравнений}

Решение СЛАУ методом Гаусса. Прямой ход.

Расширенная матрица СЛАУ:
\begin{equation*}
    \begin{gmatrix}[v]
        2 & 6 & -1 & -6 \\
        5 & -1 & 2 & 35 \\
        -3 & -4 & 1 & 11
    \end{gmatrix}.
\end{equation*}

Преобразование матрицы:
\begin{align*}
    \begin{gmatrix}[v]
        2 & 6 & -1 & -6 \\
        5 & -1 & 2 & 35 \\
        -3 & -4 & 1 & 11
        \rowops
            \mult{0}{\times 5} % умножить 1 строку на 5
            \mult{1}{\times (-2)} % умножить 2 строку на -2
            \add{1}{0} % прибавить 2 строку к 1
    \end{gmatrix}
    & \longrightarrow
    \begin{gmatrix}[v]
        0 & 32 & -9 & -100 \\
        5 & -1 & 2 & 35 \\
        -3 & -4 & 1 & 11
        \rowops
            \mult{1}{\times 3} % умножить 2 строку на 3
            \mult{2}{\times 5} % умножить 3 строку на 5
            \add{2}{1} % прибавить 3 строку к 2
    \end{gmatrix} 
    \longrightarrow{} \\ \longrightarrow{}
    \begin{gmatrix}[v]
        0 & 32 & -9 & -100 \\
        0 & -23 & 11 & 160 \\
        -3 & -4 & 1 & 11
        \rowops
            \mult{0}{\times 23} % умножить 1 строку на 23
            \mult{1}{\times 32} % умножить 2 строку на 32
            \add{1}{0} % прибавить 2 строку к 1
    \end{gmatrix}
    & \longrightarrow
    \begin{gmatrix}[v]
            0 & 0 & 146 & 2820 \\
            0 & -23 & 11 & 160 \\
            -3 & -4 & 1 & 11
        \end{gmatrix}.
\end{align*}

Обратный ход:
\begin{align*}
    x_3 &= \frac{2820}{145}=\frac{564}{29}, \\
    x_2 &= \frac{160-11x_3}{-23}=\frac{160-11\times\frac{564}{29}}{-23}=\frac{68}{29}, \\
    x_1 &= \frac{11+4x_2-x_3}{-3}=\frac{11+4\times\frac{68}{29}-\frac{568}{29}}{-3}=-\frac{9}{29}.
\end{align*}

\subsubsection*{Приближенное решение системы уравнений}

Решение СЛАУ методом Зейделя.

\begin{equation*}
    \begin{cases}
        $$x_1=7+\frac{1}{5}x_1+\frac{1}{6}x_3$$, \\
        $$x_2=-1-\frac{1}{3}x_1+\frac{1}{6}x_3$$, \\
        $$x_3=11+3x_1+4x_2$$.
    \end{cases}
\end{equation*}

Примем за~начальное приближение:
\begin{equation*}
    \begin{cases}
        $$x_1^{\{0\}}=0$$, \\
        $$x_2^{\{0\}}=0$$, \\
        $$x_3^{\{0\}}=0$$.
    \end{cases}
\end{equation*}

См.~таблицу~\ref{sr1Task3Table} на~стр.~\pageref{sr1Task3Table}.

\begin{table}[htb]
    \centering
    \begin{tabular}{|*{3}{S[round-precision=8]|}}
        \toprule
        
        {$x_1^{\{m\}}$} & {$x_2^{\{m\}}$} & {$x_3^{\{m\}}$} \\
        
        \midrule
        
        0 & 0 & 0 \\
        
        7 & -3,333333333 & 18,66666667 \\
        
        -1,133333333 & 2,488888889 & 17,55555556 \\
        
        0,4755555556 & 1,767407407 & 19,4962963 \\
        
        -0,445037037 & 2,397728395 & 19,25580247 \\
        
        -0,2227753086 & 2,283558848 & 19,46590947 \\
        
        -0,3296520165 & 2,35420225 & 19,42785295 \\
        
        -0,3003007298 & 2,338075735 & 19,45140075 \\
        
        -0,312945153 & 2,346215176 & 19,44602524 \\
        
        -0,3091670628 & 2,344059895 & 19,44873839 \\
        
        -0,3106833778 & 2,345017525 & 19,44801997 \\
        
        -0,3102044811 & 2,344738155 & 19,44833917 \\
        
        -0,310388039 & 2,344852542 & 19,44824605 \\
        
        -0,3103279122 & 2,344816979 & 19,44828418 \\
        
        \bottomrule
    \end{tabular}
    \caption{Результаты приближения методом Зейделя}
    \label{sr1Task3Table}
\end{table}

Функции\footnote{представлены формулы для~строки~3} электронной таблицы\footnote{Google Sheets}~\ref{sr1Task3Table}:
\begin{description}
    \item[столбец A, кроме первой итерации] \verb"=7+B2/5-C2*2/5";
    
    \item[столбец B, кроме первой итерации]  \verb"=-1-A3/3+C2/6";
    
    \item[столбец C, кроме первой итерации] \verb"=11+3*A3+4*B3".
    
\end{description}

\begin{align*}
    \left|x_1^{\{13\}}-x_1^{\{12\}}\right| &< 10^{-4}, \\
    \left|x_2^{\{13\}}-x_2^{\{12\}}\right| &< 10^{-4}, \\
    \left|x_3^{\{13\}}-x_3^{\{12\}}\right| &< 10^{-4}.
\end{align*}

При $\varepsilon=10^{-4}$:
\begin{equation*}
    \begin{cases}
        $$x_1=0,31032$$, \\
        $$x_2=2,34481$$, \\
        $$x_3=19,44828$$.
    \end{cases}
\end{equation*}

\clearpage
\section{Самостоятельная работа \textnumero~2. Построение интерполяционных многочленов}

\subsection*{Приближение функции}

Необходимо найти приближение функции, заданной в~точках, многочленом, значения которого совпадают со~значениями функции в~указанных точках:

\begin{center}
	\begin{tabular}{SS}
		\toprule
		
		$x$ & $y$ \\
		
		\midrule
		
		1 & 6 \\
		3 & 10 \\
		5 & 8 \\
		7 & 12 \\
		9 & 14 \\
		
		\bottomrule
	\end{tabular}
\end{center}

См.~таблицу~\ref{sr2Table} на~стр.~\pageref{sr2Table}.

\begin{table}[htb]
    \centering
    \begin{tabular}{|*{6}{c|}}
         \toprule
         
         $x$ & $y$ & $\Delta y$ & $\Delta^2 y$ & $\Delta^3 y$ & $\Delta^4 y$  \\
         
         \midrule
         
         \begin{tabular}{S}
              1  \\ 3 \\ 5 \\ 7 \\ 9
         \end{tabular}
         &
         \begin{tabular}{S}
              6 \\ 10 \\ 8 \\ 12 \\ 14
         \end{tabular}
         &
         \begin{tabular}{S}
              4 \\ -2 \\ 4 \\ 2
         \end{tabular}
         &
         \begin{tabular}{S}
              -6 \\ 6 \\ -2
         \end{tabular}
         &
         \begin{tabular}{S}
              12 \\ -8
         \end{tabular}
         &
         \begin{tabular}{S}
         	-20
         \end{tabular}
         \\ 
         
         \bottomrule
    \end{tabular}
    \caption{Расчёт разностей табличной функции}
    \label{sr2Table}
\end{table}

Узлы интерполяции равноотстоящие, т.к. $h = 3-1 = 5-3 = 7-5 \hm = 9-7 = 2 = const$.

Первая интерполяционная формула Ньютона:
\begin{multline*}
    P_n(x) = 
    y_0 +
    \frac{\Delta y_0}{1!h}(x-x_0) +
    \frac{\Delta^2 y_0}{2!h^2}(x-x_0)(x-x_1) +
    \ldots + \\ +
    \frac{\Delta^n y_0}{n!h^n}(x-x_0)(x-x_1)\ldots(x-x_{n-1}) .
\end{multline*}

\begin{multline*}
    P_n(x) =
    6 +
    \frac{4}{1!2}(x-1) +
    \frac{-6}{2!2^2}(x-1)(x-3) + \\ +
    \frac{12}{3!2^3}(x-1)(x-3)(x-5) + \\ +
    \frac{-20}{4!2^4}(x-1)(x-3)(x-5)(x-7) = \\ =
    \frac{-5x^4+104x^3-718x^2+1912x-717}{96} = \\ =
    -\frac{5}{96}x^4+\frac{13}{12}x^3-\frac{359}{48}x^2+\frac{239}{12}x-\frac{239}{32} .
\end{multline*}

\subsection*{Построение графика}

См.~рис.~\ref{sr2Function} на~стр.~\pageref{sr2Function}.

\begin{figure}[htb]
    \centering
    \begin{tikzpicture}
        \begin{axis}[legend pos = south east, axis lines = left, grid = both, xlabel = $x$, ylabel = {$f(x)$}, ymin=5, xmin=0]
            \addplot [domain=0:10, samples=100, color=red]{(-5*x^4+104*x^3-718*x^2+1912*x-717)/96};
            \addlegendentry{\(-\frac{5}{96}x^4+\frac{13}{12}x^3-\frac{359}{48}x^2+\frac{239}{12}x-\frac{239}{32}\)};
            \addplot[mark=*] coordinates {(1,6)} node [above right,pos=1] {$(1;\,6)$};
            \addplot[mark=*] coordinates {(3,10)} node [right,pos=1] {$(3;\,10)$};
            \addplot[mark=*] coordinates {(5,8)} node [below,pos=1] {$(5;\,8)$};
            \addplot[mark=*] coordinates {(7,12)} node [right,pos=1] {$(7;\,12)$};
            \addplot[mark=*] coordinates {(9,14)} node [below left,pos=1] {$(9;\,14)$};
        \end{axis}
    \end{tikzpicture}
    \caption{График полученного интерполяционного многочлена}
    \label{sr2Function}
\end{figure}

\subsection*{Значение в~точке}

Найти значение в~точке $x=6$.

\begin{multline*}
    P_n(6) =
    -\frac{5}{96}\times6^4+\frac{13}{12}\times6^3-\frac{359}{48}\times6^2+\frac{239}{12}\times6-\frac{239}{32}= \\ =
    -\frac{135}{2}+234-\frac{1077}{4}+\frac{239}{2}-\frac{239}{32}=\frac{297}{32}.
\end{multline*}

\clearpage
\section{Самостоятельная работа \textnumero~3. Вычисление определённых интегралов}

\subsection*{Аналитическое выражение}

Найти аналитическое выражение для~неопределённого интеграла $\int \sin(\ln(6x))\,dx$.

\begin{multline*}
    \int\sin(\ln(6x))\,dx =
    \left[u=6x, \quad du=6dx\right] =
    \frac{1}{6}\int\sin(\ln(u))\,du = \\ =
    \left[v=\ln(u), \quad dv=\frac{1}{u}\,du\right] =
    \frac{1}{6}\int\exp(v)\sin(v)\,dv = \\ =
    \frac{1}{6}\left(\frac{1}{2}\exp(v)((\sin(v)-\cos(v))\right)+C = \\ =
    \frac{1}{12}u(\sin(\ln(u))-\cos(\ln(u)))+C = \\ =
    \frac{1}{2}x(\sin(\ln(6x))-\cos(\ln(6x)))+C.
\end{multline*}

\subsection*{Графики интеграла и подынтегральной функции}

Построить графики найденного интеграла~--- красным цветом и подынтегральной функции~--- синим цветом.

См.~рис.~\ref{sr3Functions} на~стр.~\pageref{sr3Functions}.

\begin{figure}[htb]
    \begin{tikzpicture}
        \centering
        \begin{axis}[legend pos = south west, axis lines = left, grid = both, xlabel = $x$, ylabel = {$f(x)$}, ymax=.2, ymin=-1, domain=.0001:2]
            \addplot [samples=20, color=red]{1/2*x*(sin(ln(6*x))-cos(ln(6*x)))};
            \addlegendentry{\(\frac{1}{2}x(\sin(\ln(6x))-\cos(\ln(6x)))+0\)};
            \addplot [samples=200, color=blue]{sin(ln(6*x))};
            \addlegendentry{\(\sin(\ln(6x))\)};
        \end{axis}
    \end{tikzpicture}
    \caption{Графики интеграла, подынтегральной функции}
    \label{sr3Functions}
\end{figure}


\subsection*{Приближенное значение интеграла}

Вычислить приближенное значение интеграла: $$\int\limits_2^{8} \sin(\ln(6x))\,dx .$$

Формула Симпсона:
$$\int\limits_a^b f(x) \, dx = \frac{b-a}{6}\left(f(a)+4f\left(\frac{a+b}{2}\right)+f(b)\right) .$$

\begin{gather*}
    \begin{aligned}
        f(2)&=\sin(\ln(6\times2))=0,611 ,\\
        f\left(\frac{2+8}{2}\right)&=\sin(\ln(6\times5))=-0,257 ,\\
        f(8)&=\sin(\ln(6\times8))=-0,667 .
    \end{aligned}
    \\
    \int\limits_2^{8} \sin(\ln(6x))\,dx = \frac{8-2}{6}(0,611+4(-0,257)-0,667)=-1,083 .
\end{gather*}

Ошибка аппроксимации:
$$E=-\frac{1}{90} \left(\frac{b-a}{2}\right)^5 \max(f^{(4)}(\xi)) \text{, \quad где } \xi\in[a; b] .$$

\begin{align*}
    f'(x) &= \frac{\cos(\ln(6)+\ln(x))}{x} ,\\
    f''(x) &= - \frac{\sin(\ln(6)+\ln(x))+\cos(\ln(6)+\ln(x))}{x^2} ,\\
    f'''(x) &= \frac{\cos(\ln(6)+\ln(x))+3\sin(\ln(6)+\ln(x))}{x^3} ,\\
    f^{(4)}(x) &= - \frac{10\sin(\ln(6)+\ln(x))}{x^4} .
\end{align*}

Нахождение $\max(f^{(4)}(x))$ на отрезке $x=[a; b]$:
\begin{gather*}
    f^{(5)}(x) = \frac{40\sin(\ln(6)+\ln(x))-10\cos(\ln(6)+\ln(x))}{x^5} .\\
    \frac{40\sin(\ln(6x))-10\cos(\ln(6x))}{x^5} = 0 ,\\
    \begin{cases}
        x_1\approx0,213 ,\\
        x_2\approx4,927 .
    \end{cases}
    \\
    \begin{aligned}
        f^{(4)}(0,213) &\approx -1,1798\times10^7 ,\\
        f^{(4)}(4,927) &\approx 0,0041 ,\\
        f^{(4)}(2) &\approx -0,38156 ,\\
        f^{(4)}(8) &\approx 0,0016 .\\
    \end{aligned}
    \\
    \max(f^{(4)}(\xi)) = f^{(4)}(4,927) \approx 0,0041 .\\
    E = -\frac{1}{90}\left(\frac{8-2}{2}\right)^5\times0,0041 \approx -0,01107 .\\
    \int\limits_2^{8} \sin(\ln(6x))\,dx = -1,083\pm0,01107 .\\
\end{gather*}


\subsection*{Приближенное значение интеграла}

Вычислить приближенное значение интеграла:
$$\int\limits_2^8 \exp(-x^2)\sin(6x)\,dx .$$

Метод трапеций:
$$\int\limits_a^b f(x) \, dx \approx \frac{\Delta}{2} (f(x_0)+2f(x_1)+2f(x_2)+\dots+2f(x_{N-1})+f(x_N) .$$

Разобьём интеграл на~$N=6$ равных промежутков. $\Delta x=\frac{b-a}{N}=\frac{8-2}{6}$.

См.~талицу~\ref{sr3Table} на~стр.~\pageref{sr3Table}.

\begin{table}[htb]
    \centering
    \begin{tabular}{|*{3}{S}}
        \toprule
        
        $i$ & $x_i$ & $y_i$ \\ 
        
        \midrule
        
        0 & 2 & -0,0098 \\
        1 & 3 & -9,27e-5 $\phantom{}\approx 0$ \\
        2 & 4 & -1,02e-7 $\phantom{}\approx 0$ \\
        3 & 5 & -1,37e-11 $\phantom{}\approx 0$ \\
        4 & 6 & -2,3e-16 $\phantom{}\approx 0$ \\
        5 & 7 & -4,81e-22 $\phantom{}\approx 0$ \\
        6 & 8 & -1,23e-28 $\phantom{}\approx 0$ \\
        
        \bottomrule
    \end{tabular}
    \caption{Результат приближения значения интеграла}
    \label{sr3Table}
\end{table}

$$\int\limits_2^8 \exp(-x^2)\sin(6x)\,dx \approx \frac{1}{2}(-0,0098)=-0,0049 .$$

Анализ ошибки:
$$E=-\frac{(b-a)^3}{12N^2}\max(f''(\xi)) \text{, \quad где } \xi\in[a; b] .$$

\begin{multline*}
    f''(x) = -38\exp(-x^2)\sin(6x)+4x^2\exp(-x^2)\sin(6x)- \\
    -24x\exp(-x^2)\cos(6x) = \\
    = 4\exp(-x^2)\left(\sin(6x)x^2-6\cos(6x)x-\frac{19\sin(6x)}{2}\right) .
\end{multline*}

Нахождение $\max(f''(x))$, где~$x\in[2; 8]$:

\begin{multline*}
    f'''(x)=228x\exp(-x^2)\sin(6x)-252\exp(-x^2)\cos(6x)- \\
    -8x^3\exp(-x^2)\sin(6x)+72x^2\exp(-x^2)\cos(6x) = \\
    = -8\exp(-x^2)\left(\sin(6x)x^3-9\cos(6x)x^2-\frac{57\sin(6x)x}{2}+\frac{63\cos(6x)}{2}\right) .
\end{multline*}

При~$f'''(x) = 0$:
\begin{gather*}
    \begin{cases}
        x_1 \approx 0,228 ,\\
        x_2 \approx 3,507 .
    \end{cases} 
    \\
    \begin{aligned}
        f''(0,228) &\approx -36,186 ,\\
        f''(3,507) &\approx 0,00026 ,\\
        f''(2) &\approx -0,5257 ,\\
        f''(8) &\approx -71,48\times10^{-28} .
    \end{aligned}
    \\
    \max(f''(x))=0,00026 .\\
    E=-\frac{(8-2)^3}{12\times1^2}\times0,00026=-0,00468 .\\
    \int\limits_2^8 \exp(-x^2)\sin(6x)\,dx \approx \frac{1}{2}(-0,0098)=-0,0049\pm0,00468 .
\end{gather*}


\clearpage
\section{Самостоятельная работа \textnumero~4. Решение обыкновенных дифференциальных уравнений}

\subsection*{Аналитическое решение задачи Коши}

Найти аналитическое решение задачи Коши:
\begin{gather*}
    y'(t)=\frac{1}{6}(t+y), \quad y(0)=6 .\\
    -\frac{y}{6}+y'=\frac{t}{6} .\\
\end{gather*}

Замена на $y=uv$, $y'=u'v+uv'$:
\begin{gather*}
    -\frac{uv}{6}+uv'+u'v=\frac{t}{6} ,\\
    u\left(-\frac{v}{6}+v'\right)+u'v = \frac{t}{6} ,\\
    \begin{cases}
        u\left(-\frac{v}{6}+v'\right)=0 ,\\
        u'v=\frac{t}{6} .
    \end{cases}
\end{gather*}

Нахождение $v$. При~$u=0$:
\begin{gather*}
    -\frac{v}{6}+v'=0 ,\\
    v'=\frac{v}{6} ,\\
    \frac{dv}{v}=\frac{1}{6}dt ,\\
    \int\frac{dv}{v}=\frac{1}{6}\int dt ,\\
    \ln(v)=\frac{t}{6} ,\\
    v=\exp\left(\frac{t}{6}\right) .
\end{gather*}

Нахождение $u$:
\begin{gather*}
    u'v=\frac{t}{6} ,\\
    u'\exp\left(\frac{t}{6}\right)=\frac{t}{6} ,\\
    u'=\frac{t\exp\left(-\frac{t}{6}\right)}{6} ,\\
    u=\int\frac{t\exp\left(-\frac{t}{6}\right)}{6}\,dt = \frac{1}{6}\int t\exp\left(-\frac{1}{6}\right)\,dt .
\end{gather*}

Формула интегрирования по~частям:
$$\int U\,dV = UV - \int V \,dU .$$

\begin{gather*}
    \begin{aligned}
        U=t , &\quad dV=\exp\left(-\frac{t}{6}\right)dt ,\\
        dU=dt , &\quad V=-6\exp\left(-\frac{t}{6}\right) ,
    \end{aligned} 
    \\
    \begin{multlined}
        \int t\exp\left(-\frac{t}{6}\right)\,dt=-6t\exp\left(-\frac{t}{6}\right)-\int-6\exp\left(-\frac{t}{6}\right)\,dt = \\
        = -6t\exp\left(-\frac{t}{6}\right)+\int6\exp\left(-\frac{t}{6}\right)\,dt=-6t\exp\left(-\frac{t}{6}\right)-36\exp\left(-\frac{t}{6}\right)+C .
    \end{multlined}
    \\
    u=\frac{1}{6}\int t\exp\left(-\frac{t}{6}\right)\,dt = \frac{1}{6}(-6x-36)\exp\left(-\frac{t}{6}\right) ,\\
    y=uv=\left(C+(-6t-36)\frac{\exp\left(-\frac{t}{6}\right)}{6}\right)\exp\left(\frac{t}{6}\right)=C\exp\left(\frac{t}{6}\right)-t-6 .
\end{gather*}

При~$y(0)=6$:
\begin{gather*}
    6 = C\exp(0)-0-6 ,\\
    C = 12 ,\\
    y = 12\exp\left(\frac{t}{6}\right)-t-6 .
\end{gather*}


\subsection*{График найденного решения}

Построить график найденного решения на~отрезке $[0;\,6]$.

См.~рис.~\ref{sr4function1} на~стр.~\pageref{sr4function1}.

\begin{figure}[htb]
    \centering
    \begin{tikzpicture}
        \begin{axis}[legend pos = north west, axis lines = left, grid = both, xlabel = $x$, ylabel = {$f(x)$}]
            \addplot [domain=0:6, samples=30, color=red]{12*exp(x/6)-x-6};
            \addlegendentry{\(12\exp\left(\frac{x}{6}\right)-x-6\)};
        \end{axis}
    \end{tikzpicture}
    \caption{График найденного решения}
    \label{sr4function1}
\end{figure}

\subsection*{Численное решение задачи Коши}

Найти численное решение задачи Коши: 
$$y'(t)=\frac{1}{6}(t+y), \quad y(0)=6 .$$

Метод Эйлера:
$$y_{n+1}=y_n+hf(t_n,\,y_n) .$$

Пусть $h=1$, $t\in[0;\,6]$.

\begin{gather*}
    f(t,\,y)=\frac{1}{6}(t+y) ,\\
    y_{n+1}=y_n+\frac{1}{6}(t_n+y_n) ,\\
    t_0=0 , \quad y_0=6 .\\
\end{gather*}

См.~таблицу~\ref{sr4Table} на~стр.~\pageref{sr4Table}.

\begin{table}[htb]
    \centering
    \begin{tabular}{|*{3}{S|}}
    	\toprule
        
        $n$ & $t_n$ & $y_n$ \\
        
        \midrule
        
        0 & 0 & 6 \\
        1 & 1 & 7 \\
        2 & 2 & 8,333 \\
        3 & 3 & 10,055 \\
        4 & 4 & 12,231 \\
        5 & 5 & 14,936 \\
        6 & 6 & 18,259 \\
        
        \bottomrule
    \end{tabular}
    \caption{Результат численного решения}
    \label{sr4Table}
\end{table}

\subsection*{Численное решение задачи Коши}

Найти численное решение задачи Коши: $$y'(t)=\sin(6y(t)+t^2), \quad y(0)=6$$ в точках $t=1$ и $t=2$.

Метод Эйлера:
$$y_{n+1}=y_n+hf(t_n,\,y_n) .$$

Пусть $h=1$, $t\in[0;\,5]$.

\begin{gather*}
    f(t,\,y)=\sin(6y+t^2) ,\\
    y_{n+1}=y_n+\sin(6y+t^2) ,\\
    t_0=0 , \quad y_0=6 . \\
\end{gather*}

См.~таблицу~\ref{sr4Table2} на~стр.~\pageref{sr4Table2}.

\begin{table}[htb]
    \centering
    \begin{tabular}{|*{3}{S|}}
    	\toprule
    	
        $n$ & $t_n$ & $y_n$ \\
        
        \midrule
        
        0 & 0 & 6 \\
        1 & 1 & 5,008 \\
        2 & 2 & 4,648 \\
        3 & 3 & 5,103 \\
        4 & 4 & 6,043 \\
        5 & 5 & 6,955 \\
        
        \bottomrule
    \end{tabular}
    \caption{Результат численного решения}
    \label{sr4Table2}
\end{table}

$$y(1)=5,008 , \quad y(2)=4,648 .$$

\subsection*{График найденного решения}

Построить график найденного решения на~отрезке $[0;\,5]$.

См.~рис.~\ref{sr4function2} на~стр.~\pageref{sr4function2}.

\begin{figure}[htb]
    \centering
    \begin{tikzpicture}
        \begin{axis}[axis lines = left, grid = both, xlabel = $x$, ylabel = {$f(x)$}]
            \addplot[color=red, mark=*] coordinates {(0,6) (1,5.008) (2,4.648) (3,5.103) (4,6.043) (5,6.955)};
        \end{axis}
    \end{tikzpicture}
    \caption{График найденного решения}
    \label{sr4function2}
\end{figure}



\end{document}
